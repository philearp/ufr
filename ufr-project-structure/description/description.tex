\documentclass[a4paper]{article}
\usepackage{dirtree}
\usepackage{fancyhdr}

\pagestyle{fancy}  % sets fancyhdr
\rhead{Version 1.2}

\begin{document}

\section*{Project Description}

A simple heirarchy is defined to allow organisation of the UFR data.

\dirtree{%
.1 Projects.
.2 Experiments.
.3 Samples \textit{(Tests \& Pulses)}.
}

\begin{itemize}
  \item{
    \textbf{Projects} are a collections of \textbf{experiments}.
    These can be used by the user to distinguish between work packages, or to separate work for different customers.
    An example project aim may be `Investigating effect of radiation dose on fatigue behaviour of Ti-6Al-4V'.
  }
  \item{
    \textbf{Experiments (S-N curve)} comprise the batch of individual \textbf{samples} that are to be tested to build an S-N curve.
    These can be used to logically group samples of a similar material, or of a similar irradation condition, to aid organisation of the data.
    An example experiment may be `S-N Curve for 1.25 dpa irradiation of Ti-6Al-4V'.
  }
  \item{
    \textbf{Samples} contain the information describing the physical sample, such as the dimensions, material, and sample preparation steps.
    Each \textbf{sample} will have finite element modelling performed to relate sample deflection to stress.
    Each \textbf{sample} can undergo one or more \textbf{tests}, although most samples will only undergo a single test if the sample is taken to failure.
  }
  \item{
    \textbf{Tests} are the lowest level of the heirarchy that the user will interact with. The user will define the test type and the test parameters (e.g\ max.\ number of cycles, amplitude setpoint).
    This is currently implemented in LabVIEW as a `Programmed Vibration'.
    A \textbf{test} can either be a `step test', where a sequence of \textbf{pulses} are performed at ever-increasing amplitude, or a `constant-amplitude test', where a single \textbf{pulse} is given.
    A tilt-calibration is performed at the start of each test.
    Each \textbf{test} provides a single point on the S-N curve.
  }
  \item{
    A \textbf{pulse} is the application of a constant-amplitude vibration to a sample.
    Each pulse begins with a seek.
    During a pulse, QPD data, Dino-Lite images and ultrasonic data are saved.
    \textbf{Pulses} are not directly created by the user and are initiated automatically by the UFR software after the user starts a \textbf{test}.
  }  
\end{itemize}

\newpage

\section*{Example folder structure}
Names of directories are appended with a forward-slash (/).
\newline

\dirtree{%
.1 Proj\_<project-name>/.
.2 Expt\_<experiment-name>/.
.3 Samp\_<sample-name>/.
.4 <project-name>\_<experiment-name>\_<sample-name>\newline\_test-<id>\_pulse-<id>\_<datetime>\_(0001).csv/.
.4 images/.
.5 <project-name>\_<experiment-name>\_<sample-name>\newline\_test-<id>\_pulse-<id>\_<datetime>\_(0001).jpeg/.
.4 test-<id>-parameters.ufrprog.
.4 seek/.
.5 <project-name>\_<experiment-name>\_<sample-name>\newline\_test-<id>\_pulse-<id>\_<datetime>\_seek.txt.
.4 test-log.txt.
.4 calibration/.
.5 calibration-dataset.csv.
.5 calibration-parameters.csv.
.4 sample-prep/.
.4 finite-element/.
.4 Samp\_<sample-name>.info.
.3 sn-curves/.
.3 Expt\_<experiment-name>.info.
.2 Proj\_<project-name>.info.
}




\end{document}