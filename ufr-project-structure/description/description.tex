\documentclass{article}
\usepackage{dirtree}

\begin{document}

\section{Example Folder Structure}

Example folder structure:

\dirtree{%
.1 project-name/.
.2 experiment-name/.
.3 sample-name/ \textit{samples undergo test(s)}.
.4 test-id/ \textit{step-test or constant-amplitude}.
.5 pulse-id/.
.6 images/.
.6 qpd-data/.
.6 ultrasonic-data/.
.6 pulse-summary/.
.6 pulse-parameters.txt.
.6 pulse-log.txt.
.6 seek-info.txt.
.5 test-parameters.txt.
.5 test-log.txt.
.5 test-summary/.
.5 calibration/.
.4 sample-prep/.
.4 finite-element/.
.4 sample-info.txt.
.3 sn-curves/.
.3 experiment-info.txt.
.2 project-info.txt.
}

\begin{itemize}
  \item{
    \textbf{Projects} are a collections of \textbf{experiments}.
    An example project may be `RaDIATE Titanium 2020`
  }
  \item{
    \textbf{Experiments} comprise the batch of individual \textbf{samples} that are to be tested.
    These can be used to logically group samples of a similar material, or of a similar irradation condition, to aid organisation of the data.
    An example experiment may be `Irradiation dose comparison on Ti-6Al-4V'.
  }
  \item{
    \textbf{Samples} contain the information describing the physical sample, such as the dimensions, material, and sample preparation steps.
    Each \textbf{sample} will have finite element modelling performed to relate sample deflection to stress.
    Each \textbf{sample} can undergo one or more \textbf{tests}.
  }
  \item{
    A \textbf{test} can either be a `step test', where a sequence of \textbf{pulses} are performed at ever-increasing amplitude, or a `constant-amplitude test', where a single \textbf{pulse} is given.
    A tilt-calibration is performed at the start of each test.
  \item{
    A \textbf{pulse} is the application of a constant-amplitude vibration to a sample.
    This is currently implemented in LabVIEW as a `Programmed Vibration'.
    During a pulse, qpd data, Dino-Lite images and ultrasonic info are saved. 
  }
  }
\end{itemize}


\end{document}